\documentclass[12pt]{article}

\usepackage{amsmath}
\usepackage{amssymb}
\usepackage{amsthm}
\usepackage[pdftex]{graphicx}
\usepackage{setspace}
\usepackage{caption}
\usepackage{subcaption}
\usepackage{float}
\usepackage[margin=1in]{geometry}
\usepackage{listings}
\usepackage{textcomp}
\usepackage{multicol}
\usepackage[toc,page]{appendix}
\usepackage{listings}
\usepackage{fancyvrb}
\usepackage{hyperref}
\usepackage{lstbayes}

\usepackage[usenames,dvipsnames]{color}
\definecolor{DGrey}{gray}{0.25}
\definecolor{MGrey}{gray}{0.50}
\definecolor{LGrey}{gray}{0.75}

\usepackage[ruled,vlined,linesnumbered]{algorithm2e}
\newcommand\mycommfont[1]{\footnotesize\ttfamily\textcolor{Gray}{#1}}
\SetCommentSty{mycommfont}

\usepackage{inconsolata}

\usepackage[parfill]{parskip}
\setlength{\parindent}{0pt}
\setlength{\parskip}{\baselineskip}

\newcommand{\et}{e^{i\theta}}
\newcommand{\oo}{\mathcal{O}}
\newcommand{\skipline}{\bigskip\bigskip\bigskip}

\lstdefinestyle{Rsty} { 
    language=R,                         % the language of the code
    basicstyle=\footnotesize\ttfamily,  % the size of the fonts that are used for the code
    numbers=left,                       % where to put the line-numbers
    numberstyle=\footnotesize\color{LGrey},      % the style that is used for the line-numbers
    stepnumber=1,                       % the step between two line-numbers. If it is 1, each line
                                        % will be numbered
    numbersep=5pt,                      % how far the line-numbers are from the code
    backgroundcolor=\color{white},      % choose the background color. You must add \usepackage{color}
    showspaces=false,                   % show spaces adding particular underscores
    showstringspaces=false,             % underline spaces within strings
    showtabs=false,                     % show tabs within strings adding particular underscores
    frame=single,                       % adds a frame around the code
    rulecolor=\color{black},            % if not set, the frame-color may be changed on line-breaks within not-black text (e.g. commens (green here))
    tabsize=2,                          % sets default tabsize to 2 spaces
    captionpos=b,                       % sets the caption-position to bottom
    breaklines=true,                    % sets automatic line breaking
    breakatwhitespace=false,            % sets if automatic breaks should only happen at whitespace
    keywordstyle=\color{DGrey},     % keyword style
    commentstyle=\color{LGrey},   % comment style
    stringstyle=\color{MGrey},    % string literal style
    literate={<-}{{$\gets$}}1,           % prettier assignment arrows
    xleftmargin=4.0ex,
    deletekeywords={I,density,rect,_,palette,data,scale,panel,R,frame,labels,options}
}

\lstnewenvironment{R}
{\lstset{style=Rsty}}
{}

\lstdefinestyle{Cppsty} { 
    language=C++,                         % the language of the code
    basicstyle=\footnotesize\ttfamily,  % the size of the fonts that are used for the code
    numbers=left,                       % where to put the line-numbers
    numberstyle=\footnotesize\color{LGrey},      % the style that is used for the line-numbers
    stepnumber=1,                       % the step between two line-numbers. If it is 1, each line
                                        % will be numbered
    numbersep=5pt,                      % how far the line-numbers are from the code
    backgroundcolor=\color{white},      % choose the background color. You must add \usepackage{color}
    showspaces=false,                   % show spaces adding particular underscores
    showstringspaces=false,             % underline spaces within strings
    showtabs=false,                     % show tabs within strings adding particular underscores
    frame=single,                       % adds a frame around the code
    rulecolor=\color{black},            % if not set, the frame-color may be changed on line-breaks within not-black text (e.g. commens (green here))
    tabsize=4,                          % sets default tabsize to 2 spaces
    captionpos=b,                       % sets the caption-position to bottom
    breaklines=true,                    % sets automatic line breaking
    breakatwhitespace=false,            % sets if automatic breaks should only happen at whitespace
    keywordstyle=\color{DGrey},     % keyword style
    commentstyle=\color{LGrey},   % comment style
    stringstyle=\color{MGrey},    % string literal style
    %literate={<-}{{$\gets$}}1,           % prettier assignment arrows
    xleftmargin=4.0ex,
    deletekeywords={T}
}

\lstnewenvironment{CPP}
{\lstset{style=Cppsty}}
{}

\lstdefinestyle{Stansty} { 
    language=Stan,                         % the language of the code
    basicstyle=\footnotesize\ttfamily,  % the size of the fonts that are used for the code
    numbers=left,                       % where to put the line-numbers
    numberstyle=\footnotesize\color{LGrey},      % the style that is used for the line-numbers
    stepnumber=1,                       % the step between two line-numbers. If it is 1, each line
                                        % will be numbered
    numbersep=5pt,                      % how far the line-numbers are from the code
    backgroundcolor=\color{white},      % choose the background color. You must add \usepackage{color}
    showspaces=false,                   % show spaces adding particular underscores
    showstringspaces=false,             % underline spaces within strings
    showtabs=false,                     % show tabs within strings adding particular underscores
    frame=single,                       % adds a frame around the code
    rulecolor=\color{black},            % if not set, the frame-color may be changed on line-breaks within not-black text (e.g. commens (green here))
    tabsize=2,                          % sets default tabsize to 2 spaces
    captionpos=b,                       % sets the caption-position to bottom
    breaklines=true,                    % sets automatic line breaking
    breakatwhitespace=false,            % sets if automatic breaks should only happen at whitespace
    keywordstyle=\color{DGrey},     % keyword style
    commentstyle=\color{LGrey},   % comment style
    stringstyle=\color{MGrey},    % string literal style
    %literate={<-}{{$\gets$}}1,           % prettier assignment arrows
    xleftmargin=4.0ex,
    deletekeywords={T}
}

\lstnewenvironment{Stan}
{\lstset{style=Stansty}}
{}

\renewcommand{\arraystretch}{2}

\begin{document}

\noindent
{\LARGE {\bf SMAP and SIRS} }
\\\\
Dexter Barrows\\
\today

\section{S-maps}

    A family of forecasting methods that shy away from the mechanistic model-based approaches outlined in the previous sections have been developed by Sugihara (references) over the last several decades. As these methods do not include a mechanistic model in their forecasting process, they also do not attempt to perform parameter inference. Instead they attempt to reconstruct the underlying dynamical process as a weighted linear model from a time series.

    One such method, the sequential locally weighted global linear maps (S-map), builds a global linear map model and uses it to produce forecasts directly. Despite relying on a linear mapping, the S-map does not assume the time series on which it is operating is the product of linear system dynamics, and in fact was developed to accommodate non-linear dynamics.

    The S-map works by first constructing a time series embedding of length $E$, known as the library and denoted $\{\mathbf{x_i}\}$. Consider a time series of length $T$ denoted $x_1, x_2,..., x_T$. Each element in the time series with indices in the range $E,E+1,...,T$ will have a corresponding entry in the library such that a given element $x_t$ will correspond to a library vector of the form $\mathbf{x_i} = (x_t, x_{t-1},...,x_{t-E+1})$. Next, given a forecast length $L$ (representing $L$ time steps into the future), each library vector $\mathbf{x_i}$ is assigned a prediction from the time series $y_i = x_{t+L}$, where $x_t$ is the first entry in $\mathbf{x_i}$. Finally, a forecast ${\hat{y_t}}$ for specified predictor vector $\mathbf{x_t}$ (usually from the library itself), is generated using an exponentially weighted function of the library $\{\mathbf{x_i}\}$, predictions $\{y_i\}$, and predictor vector $\mathbf{x_t}$.

    This function is defined as follows:

    First construct a matrix $A$ and vector $b$ defined as

    \begin{equation}\label{AB}
    	\begin{array}{rl}
        \displaystyle
            A(i, j) & = w (||\mathbf{x_i}-\mathbf{x_t}||) \mathbf{x_i}(j) \\
            b(i) 	& = w (||\mathbf{x_i}-\mathbf{x_t}||) y_i
        \end{array}
    \end{equation}

    where $i$ ranges over 1 to the length of the library, and $j$ ranges over $[0,E]$. It should be noted that in the above equations and the ones that follow, $x_t(0) = 1$ to account for the linear term in the map.

	The weighting function $w$ is defined as

	\begin{equation}
		w(d) = \exp \left( \frac{-\theta d}{\bar{d}} \right) ,
	\end{equation}

	where $d$ is the euclidean distance between the predictor vector and library vectors in Equation (\ref{AB}) and $\bar{d}$ is the average of these distances. We can then see that $\theta$ serves as a way to specify the appropriate level of penalization applied to poorly-matching library vectors -- if $\theta$ is 0 all weights are the same (no penalization), and increasing $\theta$ increases the level of penalization.

	Now we solve the system $Ac = b$ to obtain the linear weightings used in to generate the forecast according to

	\begin{equation}
		\hat{y_t} = \sum_{j = 0}^{E} c_t(j) \mathbf{x_t}(j) .
	\end{equation}

	In this way we have produced a forecast value for a single time. This process can be repeated for a sequence of times $T + 1, T + 2, ...$ to project a time series into the future.


\section{S-map Algorithm}

    The above description can be summarized in algorithm

    \begin{algorithm}[H]

        \BlankLine

        \SetKwInOut{Input}{Input}
        \SetKwInOut{Output}{Output}
        \DontPrintSemicolon

        \tcc{Select a starting point}
        \Input{Time series $x_1, x_2, ..., x_T$, embedding dimension $E$, distance penalization $\theta$, forecast length $L$, predictor vector $\mathbf{x_t}$}

        \BlankLine

        \tcc{Construct library}
        \For{$i = E:T$}{
        	$\{{\mathbf{x_i}}\}$, $\mathbf{x_i} = (x_i, x_{i-1}, ...,x_{i-E-1})$
        }

        \BlankLine

        \For{$i = 1:(T_E+1)$}{
	        $b(i) = w(||\mathbf{x_i} - \mathbf{x_t}||) y_i$
        	\For{$j = 1:E$}{
	        	$A(i,j) = w(||\mathbf{x_i} - \mathbf{x_t}||) \mathbf{x_i}(j)$
	        }
        }

        \BlankLine

        \tcc{Use SVD to solve Ac = b}
        $SVD(Ac = b)$

        \BlankLine

        \tcc{Compute forecast}
        $\hat{y_t} = \sum_{j = 0}^{E} c_t (j) \mathbf{x_t} (j)$

        \BlankLine

        \tcc{Forecasted value in time series}
        \Output{Forecast $\hat{y_t}$}

        \BlankLine

        \caption{S-map}\label{smap}

    \end{algorithm}


\newpage
\begin{appendices}

	\section{SMAP Code}

		This code implements an SMAP function on a user-provided time series.

		\lstinputlisting[style=Rsty]{../../code/smap/smap.r}

	\section{RStan SIRS Code}

	    This code implements a periodic SIRS model in Rstan.

	    \lstinputlisting[style=Rsty]{../../code/hmc/hmc-sirs/sirs-euler.stan}

    \section{IF2 SIRS Code}

	    This code implements a periodic SIRS model using IF2 in C++.

	    \lstinputlisting[style=Cppsty]{../../code/if2/if2-sirs/if2-sirs.cpp}

\end{appendices}



\end{document}