%!TEX program = xelatex

\documentclass[12pt]{report}

\usepackage[utf8]{inputenc}
\usepackage{lipsum}
\usepackage{amsmath}
\usepackage{amssymb}
\usepackage{amsthm}
\usepackage{graphicx}
\usepackage{setspace}
\usepackage{caption}
\usepackage{subcaption}
\usepackage{float}
\usepackage[top=38mm,left=38mm,bottom=25mm,right=25mm]{geometry}
\usepackage{listings}
\usepackage{textcomp}
\usepackage{multicol}
\usepackage[toc,page]{appendix}
\usepackage{listings}
\usepackage{fancyvrb}
\usepackage{hyperref}
\usepackage{etoolbox}

\usepackage[displaymath, mathlines, pagewise]{lineno}
\renewcommand\linenumberfont{\normalfont\scriptsize\rmfamily}

\usepackage{lstbayes}
\usepackage{subfiles}
\usepackage{fancyhdr}
\usepackage{booktabs}
\usepackage[section]{placeins}

\pagestyle{fancy}
\fancyhead{}
\fancyhead[LE,RO]{McMaster University - Mathematics}
\fancyhead[LO,RE]{M.Sc. Thesis - Dexter Barrows}
\fancyfoot{}
\fancyfoot[C]{\thepage}
%\renewcommand{\footrulewidth}{0.4pt}% default is 0pt

%\setlength{\belowcaptionskip}{-10pt}

\usepackage[backend=biber]{biblatex}
\addbibresource{thesis-refs.bib}
\addbibresource{thesis-refs-2.bib}

\usepackage[dvipsnames]{xcolor}
\definecolor{DGrey}{gray}{0.25}
\definecolor{MGrey}{gray}{0.50}
\definecolor{LGrey}{gray}{0.75}

\hypersetup{%
  colorlinks=false,
  linkbordercolor=red
}

\usepackage[ruled,vlined,linesnumbered]{algorithm2e}
\newcommand\mycommfont[1]{\footnotesize\ttfamily\textcolor{gray}{#1}}
\SetCommentSty{mycommfont}

\usepackage{inconsolata}

\usepackage[parfill]{parskip}
\setlength{\parindent}{0pt}
%\setlength{\parskip}{\baselineskip}

\newcommand{\et}{e^{i\theta}}
\newcommand{\oo}{\mathcal{O}}
\newcommand{\skipline}{\bigskip\bigskip\bigskip}

\lstdefinestyle{Rsty} { 
    language=R,                         % the language of the code
    basicstyle=\footnotesize\ttfamily,  % the size of the fonts that are used for the code
    numbers=left,                       % where to put the line-numbers
    numberstyle=\footnotesize\color{LGrey},      % the style that is used for the line-numbers
    stepnumber=1,                       % the step between two line-numbers. If it is 1, each line
                                        % will be numbered
    numbersep=5pt,                      % how far the line-numbers are from the code
    backgroundcolor=\color{white},      % choose the background color. You must add \usepackage{color}
    showspaces=false,                   % show spaces adding particular underscores
    showstringspaces=false,             % underline spaces within strings
    showtabs=false,                     % show tabs within strings adding particular underscores
    frame=single,                       % adds a frame around the code
    rulecolor=\color{black},            % if not set, the frame-color may be changed on line-breaks within not-black text (e.g. commens (green here))
    tabsize=2,                          % sets default tabsize to 2 spaces
    captionpos=b,                       % sets the caption-position to bottom
    breaklines=true,                    % sets automatic line breaking
    breakatwhitespace=false,            % sets if automatic breaks should only happen at whitespace
    keywordstyle=\color{DGrey},     % keyword style
    commentstyle=\color{LGrey},   % comment style
    stringstyle=\color{MGrey},    % string literal style
    literate={<-}{{$\gets$}}1,           % prettier assignment arrows
    xleftmargin=4.0ex,
    deletekeywords={I,density,rect,_,palette,data,scale,panel,R,frame,labels,options}
}

\lstnewenvironment{R}
{\lstset{style=Rsty}}
{}

\lstdefinestyle{Stansty} { 
    language=Stan,                         % the language of the code
    basicstyle=\footnotesize\ttfamily,  % the size of the fonts that are used for the code
    numbers=left,                       % where to put the line-numbers
    numberstyle=\footnotesize\color{LGrey},      % the style that is used for the line-numbers
    stepnumber=1,                       % the step between two line-numbers. If it is 1, each line
                                        % will be numbered
    numbersep=5pt,                      % how far the line-numbers are from the code
    backgroundcolor=\color{white},      % choose the background color. You must add \usepackage{color}
    showspaces=false,                   % show spaces adding particular underscores
    showstringspaces=false,             % underline spaces within strings
    showtabs=false,                     % show tabs within strings adding particular underscores
    frame=single,                       % adds a frame around the code
    rulecolor=\color{black},            % if not set, the frame-color may be changed on line-breaks within not-black text (e.g. commens (green here))
    tabsize=2,                          % sets default tabsize to 2 spaces
    captionpos=b,                       % sets the caption-position to bottom
    breaklines=true,                    % sets automatic line breaking
    breakatwhitespace=false,            % sets if automatic breaks should only happen at whitespace
    keywordstyle=\color{DGrey},     % keyword style
    commentstyle=\color{LGrey},   % comment style
    stringstyle=\color{MGrey},    % string literal style
    %literate={<-}{{$\gets$}}1,           % prettier assignment arrows
    xleftmargin=4.0ex,
    deletekeywords={T}
}

\lstnewenvironment{Stan}
{\lstset{style=Stansty}}
{}

\lstdefinestyle{Cppsty} { 
    language=C++,                         % the language of the code
    basicstyle=\footnotesize\ttfamily,  % the size of the fonts that are used for the code
    numbers=left,                       % where to put the line-numbers
    numberstyle=\footnotesize\color{LGrey},      % the style that is used for the line-numbers
    stepnumber=1,                       % the step between two line-numbers. If it is 1, each line
                                        % will be numbered
    numbersep=5pt,                      % how far the line-numbers are from the code
    backgroundcolor=\color{white},      % choose the background color. You must add \usepackage{color}
    showspaces=false,                   % show spaces adding particular underscores
    showstringspaces=false,             % underline spaces within strings
    showtabs=false,                     % show tabs within strings adding particular underscores
    frame=single,                       % adds a frame around the code
    rulecolor=\color{black},            % if not set, the frame-color may be changed on line-breaks within not-black text (e.g. commens (green here))
    tabsize=4,                          % sets default tabsize to 2 spaces
    captionpos=b,                       % sets the caption-position to bottom
    breaklines=true,                    % sets automatic line breaking
    breakatwhitespace=false,            % sets if automatic breaks should only happen at whitespace
    keywordstyle=\color{DGrey},     % keyword style
    commentstyle=\color{LGrey},   % comment style
    stringstyle=\color{MGrey},    % string literal style
    %literate={<-}{{$\gets$}}1,           % prettier assignment arrows
    xleftmargin=4.0ex,
    deletekeywords={T}
}

\lstnewenvironment{CPP}
{\lstset{style=Cppsty}}
{}

\makeatletter
\let\org@subfile\subfile
\renewcommand*{\subfile}[1]{%
  \filename@parse{#1}% LaTeX's file name parser
  \expandafter
  \graphicspath\expandafter{\expandafter{\filename@area}}%
  \org@subfile{#1}%
}
\makeatother

\renewcommand{\arraystretch}{2}

\begin{document}

	\subfile{halftitle}

	%% --------------------------------------------------------------------

	\subfile{titlepage}

	\setcounter{page}{1}
	\pagenumbering{roman}

	%% --------------------------------------------------------------------

	\subfile{infopage}

	%% --------------------------------------------------------------------

	\chapter*{Abstract}

		Forecasting tools play an important role in public response to epidemics. Despite this, limited work has been done in comparing best-in-class techniques across the broad spectrum of time series forecasting methodologies. Forecasting frameworks were developed that utilised three methods designed to work with nonlinear dynamics: Iterated Filtering (IF) 2, Hamiltonian MCMC (HMC), and S-mapping. These were compared in several forecasting scenarios including a seasonal epidemic and a spatiotemporal epidemic. IF2 combined with parametric bootstrapping produced superior predictions in all scenarios. S-mapping combined with Dewdrop Regression produced forecasts slightly less-accurate than IF2 and HMC, but demonstrated vastly reduced running times. Hence, S-mapping with or without Dewdrop Regression should be used to glean initial insight into future epidemic behaviour, while IF2 and parametric bootstrapping should be used to refine forecast estimates in time.


	\chapter*{Acknowledgements}

		There are many people I have to thank for their support over the last two years:

		My supervisor Dr.~Ben Bolker for his mentorship, advice, direction, and especially patience.

		My defence committee members Dr.~Jonathan Dushoff and Dr.~David Earn who have taken the time to read my work and provide valuable input.

		The Theobio lab for including me in stimulating discussions, even when they were over my head.

		My Mom, Dad, Joel, and Sof\'{i}a for being there for me through good times and trying ones.



	\tableofcontents

	\listoffigures

	%% --------------------------------------------------------------------
	%% --------------------------------------------------------------------

	\newpage
	
	\begin{linenumbers}

	\setcounter{page}{1}
	\pagenumbering{arabic}

	\chapter{Introduction}

		\subfile{../introduction/introduction}

	\chapter{Hamiltonian MCMC}

		\subfile{../MCMC-HMCMC/mcmc-text}

	\chapter{Iterated Filtering}

		\subfile{../PF-IF2/pf-text}

	\chapter{Parameter Fitting}

		\subfile{../SC1/sc1-text}

	\chapter{Forecasting Frameworks}

		\subfile{../SC2/sc2-text}

	\chapter{S-map and SIRS}

		\subfile{../SIRS-SMAP/sirs-smap-text}

	\chapter{Spatial Epidemics}

		\subfile{../SPATIAL/spatial-text}

	\chapter{Discussion and Future Directions}

		\subfile{../future-directions/future-directions}
	
	%% --------------------------------------------------------------------
	%% --------------------------------------------------------------------
	
	\nocite{*}
	\printbibliography

	%% --------------------------------------------------------------------
	%% --------------------------------------------------------------------

	\appendix

	\chapter{Hamiltonian MCMC}

		\subfile{../MCMC-HMCMC/mcmc-appendix}

	\chapter{Iterated Filtering}

		\subfile{../PF-IF2/pf-appendix}

	\chapter{Parameter Fitting}

		\subfile{../SC1/sc1-appendix}

	\chapter{Forecasting Frameworks}

		\subfile{../SC2/sc2-appendix}

	\chapter{S-map and SIRS}

		\subfile{../SIRS-SMAP/sirs-smap-appendix}

	\chapter{Spatial Epidemics}

		\subfile{../SPATIAL/spatial-appendix}

		\subfile{../SPATIAL/cuda-appendix}


	\end{linenumbers}
	

\end{document}

